% Options for packages loaded elsewhere
% Options for packages loaded elsewhere
\PassOptionsToPackage{unicode}{hyperref}
\PassOptionsToPackage{hyphens}{url}
\PassOptionsToPackage{dvipsnames,svgnames,x11names}{xcolor}
%
\documentclass[
]{article}
\usepackage{xcolor}
\usepackage[margin=1in]{geometry}
\usepackage{amsmath,amssymb}
\setcounter{secnumdepth}{-\maxdimen} % remove section numbering
\usepackage{iftex}
\ifPDFTeX
  \usepackage[T1]{fontenc}
  \usepackage[utf8]{inputenc}
  \usepackage{textcomp} % provide euro and other symbols
\else % if luatex or xetex
  \usepackage{unicode-math} % this also loads fontspec
  \defaultfontfeatures{Scale=MatchLowercase}
  \defaultfontfeatures[\rmfamily]{Ligatures=TeX,Scale=1}
\fi
\usepackage{lmodern}
\ifPDFTeX\else
  % xetex/luatex font selection
\fi
% Use upquote if available, for straight quotes in verbatim environments
\IfFileExists{upquote.sty}{\usepackage{upquote}}{}
\IfFileExists{microtype.sty}{% use microtype if available
  \usepackage[]{microtype}
  \UseMicrotypeSet[protrusion]{basicmath} % disable protrusion for tt fonts
}{}
\makeatletter
\@ifundefined{KOMAClassName}{% if non-KOMA class
  \IfFileExists{parskip.sty}{%
    \usepackage{parskip}
  }{% else
    \setlength{\parindent}{0pt}
    \setlength{\parskip}{6pt plus 2pt minus 1pt}}
}{% if KOMA class
  \KOMAoptions{parskip=half}}
\makeatother
% Make \paragraph and \subparagraph free-standing
\makeatletter
\ifx\paragraph\undefined\else
  \let\oldparagraph\paragraph
  \renewcommand{\paragraph}{
    \@ifstar
      \xxxParagraphStar
      \xxxParagraphNoStar
  }
  \newcommand{\xxxParagraphStar}[1]{\oldparagraph*{#1}\mbox{}}
  \newcommand{\xxxParagraphNoStar}[1]{\oldparagraph{#1}\mbox{}}
\fi
\ifx\subparagraph\undefined\else
  \let\oldsubparagraph\subparagraph
  \renewcommand{\subparagraph}{
    \@ifstar
      \xxxSubParagraphStar
      \xxxSubParagraphNoStar
  }
  \newcommand{\xxxSubParagraphStar}[1]{\oldsubparagraph*{#1}\mbox{}}
  \newcommand{\xxxSubParagraphNoStar}[1]{\oldsubparagraph{#1}\mbox{}}
\fi
\makeatother


\usepackage{longtable,booktabs,array}
\usepackage{calc} % for calculating minipage widths
% Correct order of tables after \paragraph or \subparagraph
\usepackage{etoolbox}
\makeatletter
\patchcmd\longtable{\par}{\if@noskipsec\mbox{}\fi\par}{}{}
\makeatother
% Allow footnotes in longtable head/foot
\IfFileExists{footnotehyper.sty}{\usepackage{footnotehyper}}{\usepackage{footnote}}
\makesavenoteenv{longtable}
\usepackage{graphicx}
\makeatletter
\newsavebox\pandoc@box
\newcommand*\pandocbounded[1]{% scales image to fit in text height/width
  \sbox\pandoc@box{#1}%
  \Gscale@div\@tempa{\textheight}{\dimexpr\ht\pandoc@box+\dp\pandoc@box\relax}%
  \Gscale@div\@tempb{\linewidth}{\wd\pandoc@box}%
  \ifdim\@tempb\p@<\@tempa\p@\let\@tempa\@tempb\fi% select the smaller of both
  \ifdim\@tempa\p@<\p@\scalebox{\@tempa}{\usebox\pandoc@box}%
  \else\usebox{\pandoc@box}%
  \fi%
}
% Set default figure placement to htbp
\def\fps@figure{htbp}
\makeatother





\setlength{\emergencystretch}{3em} % prevent overfull lines

\providecommand{\tightlist}{%
  \setlength{\itemsep}{0pt}\setlength{\parskip}{0pt}}



 


\makeatletter
\@ifpackageloaded{caption}{}{\usepackage{caption}}
\AtBeginDocument{%
\ifdefined\contentsname
  \renewcommand*\contentsname{Table of contents}
\else
  \newcommand\contentsname{Table of contents}
\fi
\ifdefined\listfigurename
  \renewcommand*\listfigurename{List of Figures}
\else
  \newcommand\listfigurename{List of Figures}
\fi
\ifdefined\listtablename
  \renewcommand*\listtablename{List of Tables}
\else
  \newcommand\listtablename{List of Tables}
\fi
\ifdefined\figurename
  \renewcommand*\figurename{Figure}
\else
  \newcommand\figurename{Figure}
\fi
\ifdefined\tablename
  \renewcommand*\tablename{Table}
\else
  \newcommand\tablename{Table}
\fi
}
\@ifpackageloaded{float}{}{\usepackage{float}}
\floatstyle{ruled}
\@ifundefined{c@chapter}{\newfloat{codelisting}{h}{lop}}{\newfloat{codelisting}{h}{lop}[chapter]}
\floatname{codelisting}{Listing}
\newcommand*\listoflistings{\listof{codelisting}{List of Listings}}
\makeatother
\makeatletter
\makeatother
\makeatletter
\@ifpackageloaded{caption}{}{\usepackage{caption}}
\@ifpackageloaded{subcaption}{}{\usepackage{subcaption}}
\makeatother
\usepackage{bookmark}
\IfFileExists{xurl.sty}{\usepackage{xurl}}{} % add URL line breaks if available
\urlstyle{same}
\hypersetup{
  pdftitle={Advanced Topics in Biostatistics: AI Tools for Data Science and Statistics},
  colorlinks=true,
  linkcolor={blue},
  filecolor={Maroon},
  citecolor={Blue},
  urlcolor={Blue},
  pdfcreator={LaTeX via pandoc}}


\title{Advanced Topics in Biostatistics: AI Tools for Data Science and
Statistics}
\usepackage{etoolbox}
\makeatletter
\providecommand{\subtitle}[1]{% add subtitle to \maketitle
  \apptocmd{\@title}{\par {\large #1 \par}}{}{}
}
\makeatother
\subtitle{140.850, 4th Term, 2025--2026}
\author{}
\date{2026-02-17}
\begin{document}
\maketitle


\subsection{Instructor}\label{instructor}

Erik Westlund, PhD\\
Department of Biostatistics\\
Johns Hopkins Bloomberg School of Public Health\\
\href{mailto:ewestlund@jhu.edu}{\nolinkurl{ewestlund@jhu.edu}}

\subsection{Course Description}\label{course-description}

As AI tools are rapidly adopted across research and industry, there is a
growing need for statisticians and data scientists to understand what
these tools are capable of and how to use them responsibly. This course
provides practical approaches for integrating large language models and
agent-based tools into statistical workflows. Students learn how to
structure AI-assisted processes for analysis, simulation, and pipeline
development, along with core skills in context management and agent
orchestration. The course emphasizes AI safety, privacy, and responsible
handling of sensitive data. Students will discuss these topics and do
hands-on exercises to test the strengths and weaknesses of AI tools in
practice.

\subsection{Course Details}\label{course-details}

\begin{longtable}[]{@{}ll@{}}
\toprule\noalign{}
\endhead
\bottomrule\noalign{}
\endlastfoot
\textbf{Dates} & February 19, February 26, March 5, March 12, 2026 \\
\textbf{Time} & 9:00--10:20 AM ET \\
\textbf{Location} & Zoom (link provided on CoursePlus) \\
\textbf{Format} & Lecture, live demos, discussion, hands-on exercises \\
\textbf{Grading} & Pass/Fail \\
\end{longtable}

\subsection{Course Learning
Objectives}\label{course-learning-objectives}

By the end of this course, students will be able to:

\begin{itemize}
\tightlist
\item
  Describe how large language models work at a conceptual level,
  including their capabilities and limitations for statistical work.
\item
  Evaluate the ethical implications of AI tool use in research,
  including privacy, bias, reproducibility, and academic integrity.
\item
  Use code to work with data, not the LLM itself, to protect sensitive
  and regulated data (including PHI) in privacy-sensitive contexts.
\item
  Navigate the landscape of AI tools --- chat interfaces, IDE
  integrations, CLI agents, and supporting tools --- and select
  appropriate tools for different tasks.
\item
  Use AI assistants to write, debug, and audit code for data cleaning,
  visualization, and statistical analysis.
\item
  Build analysis workflows on synthetic data and deploy validated code
  to secure data environments using Git.
\item
  Critically assess AI-generated output and identify errors,
  hallucinations, and inappropriate statistical choices.
\end{itemize}

\subsection{Schedule}\label{schedule}

\subsubsection{Session 1: Ethics and Foundations --- February
19}\label{session-1-ethics-and-foundations-february-19}

\emph{What are we dealing with, and what are the stakes?}

\begin{longtable}[]{@{}
  >{\raggedright\arraybackslash}p{(\linewidth - 4\tabcolsep) * \real{0.3684}}
  >{\raggedleft\arraybackslash}p{(\linewidth - 4\tabcolsep) * \real{0.2632}}
  >{\raggedright\arraybackslash}p{(\linewidth - 4\tabcolsep) * \real{0.3684}}@{}}
\toprule\noalign{}
\begin{minipage}[b]{\linewidth}\raggedright
Block
\end{minipage} & \begin{minipage}[b]{\linewidth}\raggedleft
Min
\end{minipage} & \begin{minipage}[b]{\linewidth}\raggedright
Topic
\end{minipage} \\
\midrule\noalign{}
\endhead
\bottomrule\noalign{}
\endlastfoot
Discussion & 15 & How are you using AI now? What feels OK, what feels
uncomfortable? \\
Lecture & 15 & AI ethics and responsible use: privacy and PHI, bias,
reproducibility, academic integrity, environmental cost --- a framework
for reasoning, not a list of rules \\
Case study & 10 & How this syllabus was built with AI --- is that OK?
Why or why not? \\
Lecture & 15 & What is an LLM? Tokens, context windows, temperature,
stochasticity. Why the same prompt gives different answers. Model
landscape: GPT, Claude, Gemini, DeepSeek \\
Hands-on & 20 & Everyone gets the same statistical task; use your own
tool; compare results across tools --- what worked, what
hallucinated? \\
Wrap-up & 5 & Introduce moonshot assignment; tools and subscriptions
overview \\
\end{longtable}

\textbf{Homework:} Brainstorm your moonshot task. Try a second AI tool
you haven't used before for a small task and note the differences.

\subsubsection{Session 2: The Toolbox --- February
26}\label{session-2-the-toolbox-february-26}

\emph{What's out there, and how do you use it?}

\begin{longtable}[]{@{}
  >{\raggedright\arraybackslash}p{(\linewidth - 4\tabcolsep) * \real{0.3684}}
  >{\raggedleft\arraybackslash}p{(\linewidth - 4\tabcolsep) * \real{0.2632}}
  >{\raggedright\arraybackslash}p{(\linewidth - 4\tabcolsep) * \real{0.3684}}@{}}
\toprule\noalign{}
\begin{minipage}[b]{\linewidth}\raggedright
Block
\end{minipage} & \begin{minipage}[b]{\linewidth}\raggedleft
Min
\end{minipage} & \begin{minipage}[b]{\linewidth}\raggedright
Topic
\end{minipage} \\
\midrule\noalign{}
\endhead
\bottomrule\noalign{}
\endlastfoot
Discussion & 10 & Debrief: what happened when you tried a new tool?
Surprises? \\
Lecture + demo & 20 & The AI tools: models and model selection, chat
windows, autocomplete tools, IDE agents (VS Code, Cursor, RStudio,
Positron), CLI agents (Claude Code, etc.), Model Context Protocol \\
Lecture + demo & 15 & The supporting tools: command line basics, tmux
for persistent sessions, Git and GitHub, API keys vs.~subscriptions,
cost management \\
Hands-on & 30 & Get set up with an IDE or CLI agent. Work through a
guided exercise: use an integrated tool (not a chat window) to complete
a coding task. \\
Wrap-up & 5 & Moonshot check-in \\
\end{longtable}

\textbf{Homework:} Make progress on your moonshot. Ensure Git is set up
and working.

\subsubsection{Session 3: AI-Assisted Statistical Workflows --- March
5}\label{session-3-ai-assisted-statistical-workflows-march-5}

\emph{Doing real work: from prompting to pipelines}

\begin{longtable}[]{@{}
  >{\raggedright\arraybackslash}p{(\linewidth - 4\tabcolsep) * \real{0.3684}}
  >{\raggedleft\arraybackslash}p{(\linewidth - 4\tabcolsep) * \real{0.2632}}
  >{\raggedright\arraybackslash}p{(\linewidth - 4\tabcolsep) * \real{0.3684}}@{}}
\toprule\noalign{}
\begin{minipage}[b]{\linewidth}\raggedright
Block
\end{minipage} & \begin{minipage}[b]{\linewidth}\raggedleft
Min
\end{minipage} & \begin{minipage}[b]{\linewidth}\raggedright
Topic
\end{minipage} \\
\midrule\noalign{}
\endhead
\bottomrule\noalign{}
\endlastfoot
Lecture + demo & 15 & Context and prompting: structuring prompts for
statistical work, project context files, system prompts, good prompt
vs.~bad prompt \\
Lecture + demo & 15 & The code-not-data principle in practice. Synthetic
data as a development strategy. Git as a bridge between AI-assisted and
secure data environments \\
Lecture & 10 & AI for statistical thinking: data cleaning, EDA,
visualization, modeling, DAGs and causal assumptions, simulation, model
validation --- where AI is a thought partner and where it confidently
misleads \\
Hands-on & 30 & Build an analysis pipeline on synthetic data using AI
tools: cleaning, summary statistics, visualization, a simple model.
Audit the output --- catch the mistakes. \\
Wrap-up & 10 & Moonshot progress reports. Prepare a 3-minute informal
share for next week. \\
\end{longtable}

\textbf{Homework:} Finish your moonshot attempt. Prepare to share what
you tried, what worked, and what didn't (3 minutes, informal, no slides
required).

\subsubsection{Session 4: Synthesis and Looking Forward --- March
12}\label{session-4-synthesis-and-looking-forward-march-12}

\emph{What did we learn, and where do we go from here?}

\begin{longtable}[]{@{}
  >{\raggedright\arraybackslash}p{(\linewidth - 4\tabcolsep) * \real{0.3684}}
  >{\raggedleft\arraybackslash}p{(\linewidth - 4\tabcolsep) * \real{0.2632}}
  >{\raggedright\arraybackslash}p{(\linewidth - 4\tabcolsep) * \real{0.3684}}@{}}
\toprule\noalign{}
\begin{minipage}[b]{\linewidth}\raggedright
Block
\end{minipage} & \begin{minipage}[b]{\linewidth}\raggedleft
Min
\end{minipage} & \begin{minipage}[b]{\linewidth}\raggedright
Topic
\end{minipage} \\
\midrule\noalign{}
\endhead
\bottomrule\noalign{}
\endlastfoot
Lecture & 10 & Reproducibility and documentation: organizing AI-assisted
projects, documenting workflows, project setup and progress tracking \\
Moonshot share & 40 & Lightning rounds: each student shares their
moonshot (\textasciitilde3 min) --- what they attempted, what worked,
what failed, what surprised them. Class discussion after each. \\
Discussion & 15 & Looking forward: picking models for different tasks,
staying current as tools change, where these tools are headed, finding
your personal line \\
Wrap-up & 15 & Course retrospective: what was most useful? What do you
wish we covered? Feedback for the pilot. \\
\end{longtable}

\subsection{Moonshot Project}\label{moonshot-project}

Each student will attempt a ``moonshot'' task using AI tools ---
something you know little about, or that feels beyond your current
capability. The goal is not to produce publication-quality work. It is
to explore the limits and possibilities of these tools in a low-stakes
way and to report honestly on what happened.

\begin{itemize}
\tightlist
\item
  \textbf{Session 1:} Brainstorm your task
\item
  \textbf{Session 3:} Work session and progress check-in
\item
  \textbf{Session 4:} Informal 3-minute presentation to the class
\end{itemize}

\subsection{Tools and Subscriptions}\label{tools-and-subscriptions}

Students should subscribe to at least one AI tool for the duration of
the course. Options include:

\begin{longtable}[]{@{}ll@{}}
\toprule\noalign{}
Tool & Cost \\
\midrule\noalign{}
\endhead
\bottomrule\noalign{}
\endlastfoot
OpenAI ChatGPT Plus & \$20/month \\
Anthropic Claude Pro & \$20/month \\
Google Gemini Advanced & \$20/month (one month free trial) \\
Cursor Pro & \$20/month \\
GitHub Copilot Pro & \$10/month \\
\end{longtable}

Variety across the class is encouraged --- we will compare how different
tools handle the same problems. If financing is a concern, please reach
out to the instructor. Google Gemini Advanced offers a free trial that
can cover the course period.

Students should also ensure that Git is installed and a GitHub account
is registered before the first class. Mac and Linux typically have Git
pre-installed. Otherwise, follow directions at
\href{https://git-scm.com/install/}{git-scm.com}.

\subsection{Course Materials}\label{course-materials}

There is no required textbook. Readings and resources will be shared on
CoursePlus.

\subsection{Methods of Assessment}\label{methods-of-assessment}

This is a pass/fail course. Assessment is based on:

\begin{itemize}
\tightlist
\item
  \textbf{Participation (50\%):} Attend all sessions and engage actively
  in discussions and hands-on exercises.
\item
  \textbf{Moonshot project (50\%):} Attempt your moonshot task and
  present your experience to the class in Session 4.
\end{itemize}

\subsection{Generative AI Policy}\label{generative-ai-policy}

Using AI tools is the subject of this course. Their use is permitted,
encouraged, and expected. It is nevertheless the student's
responsibility to understand the output of these tools and ensure their
correctness. Students are strongly encouraged to approach these tools as
learning aids and not crutches.

\subsection{Academic Ethics}\label{academic-ethics}

Students enrolled in the Bloomberg School of Public Health of The Johns
Hopkins University assume an obligation to conduct themselves in a
manner appropriate to the University's mission as an institution of
higher education. Students should be familiar with the policies and
procedures specified under Policy and Procedure Manual Student-01
(Academic Ethics) and the Student Conduct Code (Student-06), available
at
\href{https://my.publichealth.jhu.edu/Resources/PoliciesProcedures/ppm/Pages/default.aspx}{my.publichealth.jhu.edu}.

\subsection{Student Health and
Well-being}\label{student-health-and-well-being}

If you are struggling with anxiety, stress, depression, or other mental
health related concerns, please consider connecting with resources:

\begin{itemize}
\tightlist
\item
  Student support:
  \href{https://bit.ly/bsphstudentsupport}{bit.ly/bsphstudentsupport}
\item
  Mental Health Services:
  \href{https://wellbeing.jhu.edu/MentalHealthServices/}{wellbeing.jhu.edu/MentalHealthServices}
\item
  Behavioral Health Crisis Support Team (24/7): 410-516-9355
\end{itemize}

\subsection{Disability Accommodations}\label{disability-accommodations}

Student Disability Services (SDS) provides accessible and inclusive
educational experiences for students with disabilities. To request
accommodations:

\begin{enumerate}
\def\labelenumi{\arabic{enumi}.}
\tightlist
\item
  Complete the SDS online application via
  \href{https://hunter.accessiblelearning.com/JHU/ApplicationStudent.aspx}{AIM}
\item
  Submit documentation using the provided link after application
  submission
\item
  Schedule a meeting with
  \href{https://outlook.office365.com/book/SDSBSPH@live.johnshopkins.edu/}{Audrey
  Ndaba}
\end{enumerate}

More information:
\href{https://publichealth.jhu.edu/about/inclusion-diversity-anti-racism-and-equity-idare/student-disability-services}{Student
Disability Services}




\end{document}
